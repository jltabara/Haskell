\documentclass[a4paper,12pt]{article}
%%%%%%%%%%%%%%%%%%%%%%Paquetes
\usepackage[spanish]{babel}  
\usepackage{indentfirst} %%%%%%%%%%%%%%%%Crear un indent al principio
\usepackage[latin1]{inputenc}%%%%%%%%%%%%� y acentos
\usepackage{tcolorbox}
\usepackage{color}
\usepackage{mathrsfs}
\usepackage{enumerate}

\usepackage{fourier}



\usepackage[paperwidth=12cm,paperheight=15cm, hmargin={1cm,1cm}, top=1.5cm, bottom=1cm]{geometry}
\usepackage{latexsym,amsmath,amssymb,amsfonts}

\newcommand{\com}[1]{\textbf{\color{blue}{#1}}}

\pagestyle{empty}

%\newboxedtheorem[boxcolor=orange, background=blue!5, titlebackground=blue!20,
%titleboxcolor = black]{ejer}{}{anything}
\newenvironment{capitulo}{\begin{tcolorbox}[colback=red!5!white,colframe=red!75!black]}{\end{tcolorbox}\bigskip}

\newenvironment{indi}{\textbf{Indicaci�n:}}{}

\newenvironment{ejem}{\textbf{Ejemplo}: }{}

\newtcolorbox{enun}{colback=red!5!white,
colframe=red!75!black}


\newtcolorbox{ejer}[1]{colback=red!5!white,colframe=red!75!black,fonttitle=\bfseries,
title=#1}

\author{jltabara@gmail.com}

\title{Introducci�n a Mathematica}

\date{}

\begin{document}





\begin{ejer}{�rea de un c�rculo}

\begin{enun}

Una funci�n que calcule el �rea de un c�rculo de radio \textbf{r}.

\end{enun}


\begin{ejem}
\begin{verbatim}
f 6  =>  113.09733552923255
f 7.456  =>  174.64721773643396
\end{verbatim}



\end{ejem}

\begin{indi}

La f�rmula del �rea de un c�rculo es:
\[
A= \pi r^2
\]


\end{indi}

\end{ejer}

\newpage


\begin{ejer}{�rea de un tri�ngulo}

\begin{enun}

Una funci�n que calcule el �rea de un tri�ngulo conocida la base y la altura.

\end{enun}


\begin{ejem}
\begin{verbatim}
f 2 7  =>  7.0
f 8.56 9.34  =>  39.9752
\end{verbatim}
\end{ejem}

\end{ejer}

\newpage


\begin{ejer}{�rea de un tri�ngulo y f�rmula de Her�n}

\begin{enun}

Una funci�n que calcule el �rea de un tri�ngulo conocidos sus tres lados.

\end{enun}


\begin{ejem}
\begin{verbatim}
f 3 4 5  =>  6.0
f 6 9.78 4.456  =>  8.92963672556416
\end{verbatim}
\end{ejem}

\begin{indi}

Si $s$ denota el semiper�metro el �rea es:
\[
A = \sqrt{s(s-a)(s-b)(s-c)}
\]
donde $a$, $b$ y $c$ son los lados.

\end{indi}

\end{ejer}

\newpage


\begin{ejer}{�ltima cifra de un n�mero}

\begin{enun}

Una funci�n que devuelva la �ltima cifra de un n�mero.

\end{enun}

\begin{ejem}
\begin{verbatim}
f 325  =>  5
\end{verbatim}
\end{ejem}

\end{ejer}

\newpage

\begin{ejer}{Rotaci�n de listas}

\begin{enun}

Una funci�n que aplicada a una lista  coloque el primer elemento en �ltimo lugar.

\end{enun}

\begin{ejem}
\begin{verbatim}
f  [3,2,5,7]  =>  [2,5,7,3]
\end{verbatim}

\end{ejem}



\end{ejer}

\newpage


\begin{ejer}{Pal�ndromos}

\begin{enun}

Una funci�n que detecte si una cadena (o en general, una lista) es un pal�ndromo.

\end{enun}



\begin{ejem}
\begin{verbatim}
f "ana"  =>  True
f [3,2,5,2,3]  =>  True
f [3,2,5,6,2,3]  =>  False
\end{verbatim}

\end{ejem}

\end{ejer}

\newpage

\begin{ejer}{Elementos interiores de una lista}

\begin{enun}

Una funci�n que elimine el primer y �ltimo elemento de una lista.

\end{enun}

\begin{ejem}
\begin{verbatim}
f  [2,5,3,7,3]  =>  [5,3,7]
f "MazingerZ"  =>  "azinger"
\end{verbatim}

\end{ejem}

\end{ejer}



\newpage

\begin{ejer}{Desigualdad triangular}

\begin{enun}

Una funci�n que determine si tres n�meros pueden ser los lados de un tri�ngulo.

\end{enun}

\begin{ejem}
\begin{verbatim}
f  3 4 5  =>  True
f 30 4 5  =>  False
f 3 4 7  =>  False
\end{verbatim}

\end{ejem}

\begin{indi}

Tres n�meros pueden formar un tri�ngulo si la suma de cualquier par de lados es mayor que la longitud del otro lado. Esta ese la denominada \textbf{desigualdad triangular}.


\end{indi}

\end{ejer}


\newpage

\begin{ejer}{Determinaci�n del cuadrante}

\begin{enun}

Una funci�n que determine en que cuadrante est� el punto que se le pasa como argumento. El punto se puede pasar como una dupla.

\end{enun}

\begin{ejem}
\begin{verbatim}
f (3,5)  =>  1
f (-3,5)  =>  2
f (-100,-0.1)  =>  3
\end{verbatim}

\end{ejem}


\end{ejer}


\newpage

\begin{ejer}{Ecuaci�n de segundo grado}

\begin{enun}

Una funci�n que resuelva ecuaciones de segundo grado, pasando como par�metros los coeficientes del polinomio.

\end{enun}

\begin{ejem}
\begin{verbatim}
f 1 (-2) 1  =>  [1.0,1.0]
f 1 (-5) 6  =>  [2,3]
\end{verbatim}

\end{ejem}


\end{ejer}






\end{document}



